\chapter{Author Extraction}\label{cha:author-extraction}

In the following we will discuss the different steps during the author extraction process.

\section{Preprocessing}\label{sec:ae-preprocessing}

The corpus that was used in our study consists of \todo{number} research papers that are accessible via the \gls{ssoar}\footnote{\url{http://www.ssoar.info} (accessed April~21,~2016)} as \gls{pdf} documents.
Since our later steps require textual input, the first step is to extract the content from the \glspl{pdf}.
Apache PDFBox\footnote{\url{https://pdfbox.apache.org/} (accessed April~21,~2016)}, a Java library for manipulating \glspl{pdf}, allows the extraction of the content as Unicode text.
This is done by also taking into account the formatting of the document, for example when the research paper contains two text columns.
\itodo{statistics about performance}
\itodo{further preprocessing, e.g.\ scanned pdfs}

\section{Generating Training Sets with Distant Supervision}\label{sec:ae-distant-supervision}


\section{Training CRFs}\label{sec:ae-training-crfs}

