\chapter{Author Extraction}\label{cha:author-extraction}
In the following we will discuss the different steps during the author extraction process.

\section{Preprocessing}\label{sec:ae-preprocessing}

The corpus that was used in our study consists of \todo{number} research papers that are accessible via the \gls{ssoar}\footnote{\url{http://www.ssoar.info} (accessed April~21,~2016)} as \gls{pdf} documents.

Since our later steps require textual input, the first step is to extract the content from the \glspl{pdf}.
Apache PDFBox\footnote{\url{https://pdfbox.apache.org/} (accessed April~21,~2016)}, a Java library for manipulating \glspl{pdf}, allows the extraction of the content as Unicode text.
This is done by also taking into account the formatting of the document, for example when research papers contain two text columns of text.
\itodo{statistics about performance}
\itodo{reference section extraction using parscit}

\section{Generating Training Sets with Distant Supervision}\label{sec:ae-distant-supervision}

\subsection{Knowledge Base}

In order to apply distant supervision to the task of labeling author names, an external source of information is needed.
For our evaluation we use two different sources.

\bigskip

The first one is based on the \acrfull{gnd}\footnote{\url{http://www.dnb.de/EN/Standardisierung/GND/gnd.html} (accessed April~27,~2016)} which is published by the German National Library in cooperation with other library networks and institutions.
The \gls{gnd}\todo{full acr?} contains information on persons, corporate bodies, conferences and other topics with a focus on the German-speaking world which includes Germany, Austria, and Switzerland.
\itodo{which persons are included, statistics}
We will refer to the second data set as the \gls{gnd} data set.

\bigskip

The second one was extracted from the Sociological Abstracts Module\todo{true?} which is part of the \gls{csa} Title Lists\footnote{\url{http://www.proquest.com/customer-care/title-lists/Title-Lists-CSA.html} (accessed April~27,~2016)} published by the company ProQuest\footnote{\url{http://www.proquest.com/about/} (accessed April~27,~2016)}.
\itodo{statistics, refernces}
We will refer to the second data set as the \gls{csa} data set.


\subsection{Author Name Matching}
\itodo{variations of author names}
A crucial step for generating the training sets is the matching of author names in the extracted lists to their occurrences in the extracted reference sections.
The goal here is to label all possible matches as such.
This is especially interesting if possible matches overlap\todo{example}.
Usually, matches are annotated using markup languages such as XML\todo{examples}.
Since such markup language use tree based models, they don't allow a direct encoding of overlaps.
To demonstrate this issue for tree based models, we consider our previous overlap example which we want to encode using XML\@\todo{add tree example}.

There are a number of approaches that try to overcome this limitation of tree based markup languages such as \textit{milestone elements}, \textit{fragmentation}, and \textit{standoff markup}~\citep{sperberg2000goddag}. They all drastically increase the complexity of the markup document and require specialized parsers in order to retrieve data from such documents.

In order to avoid this kind of overhead, it would be preferable to annotate our author matches using a data structure that supports overlapping hierarchies by default.
\citet{sperberg2000goddag} propose such a data structure named \acrfull{goddag}.
Being a directed graph, it naturally solves the issue with overlaps by allowing a node to have multiple parents.
In order to assure a hierarchy, the graph is acyclic
In addition, it has ordered-descendants, meaning that for a given node, the order of it's child nodes is defined.
\itodo{goddag example}

%TODO decisions:
% * first name = last name
% * possible overlaps of author names

%TODO statistics





\section{Training \glsentryshortpl{crf}}\label{sec:ae-training-crfs}

