\chapter{Author Extraction}\label{cha:author-extraction}

In the following we will discuss the different steps during the author extraction process.

\section{Preprocessing}\label{sec:ae-preprocessing}

The corpus that was used in our study consists of \todo{number} research papers that are accessible via the \gls{ssoar}\footnote{\url{http://www.ssoar.info} (accessed April~21,~2016)} as \gls{pdf} documents.

Since our later steps require textual input, the first step is to extract the content from the \glspl{pdf}.
Apache PDFBox\footnote{\url{https://pdfbox.apache.org/} (accessed April~21,~2016)}, a Java library for manipulating \glspl{pdf}, allows the extraction of the content as Unicode text.
This is done by also taking into account the formatting of the document, for example when research papers contain two text columns of text.
\itodo{statistics about performance}
\itodo{reference section extraction using parscit}

\section{Generating Training Sets with Distant Supervision}\label{sec:ae-distant-supervision}

\subsection{Author Name Extraction}

In order to apply distant supervision to the task of labeling author names, an external source of information is needed.
For our evaluation we use two different sources.\\

The first one is based on the \acrfull{gnd}\footnote{\url{http://www.dnb.de/EN/Standardisierung/GND/gnd.html} (accessed April~27,~2016)} which is published by the German National Library in cooperation with other library networks and institutions.
The \gls{gnd}\todo{full acr?} contains information on persons, corporate bodies, conferences and other topics with a focus on the German-speaking world which includes Germany, Austria, and Switzerland.
\itodo{which persons are included, statistics}
We will refer to the second data set as the \gls{gnd} data set.\\

The second one was extracted from the Sociological Abstracts Module\todo{true?} which is part of the \gls{csa} Title Lists\footnote{\url{http://www.proquest.com/customer-care/title-lists/Title-Lists-CSA.html} (accessed April~27,~2016)} published by the company ProQuest\footnote{\url{http://www.proquest.com/about/} (accessed April~27,~2016)}.
\itodo{statistics, refernces}
We will refer to the second data set as the \gls{csa} data set.


\subsection{Context Extraction and Author Name Matching}
\itodo{variations of author names}
A crucial step for generating the training sets is the matching of author names in the extracted lists to their occurrences in the extracted reference sections.
%TODO why context extraction at this point
%     discuss naive approach of generating the full variations and search them
One approach is to generate a list of possible ways how author names appear in a reference string.
A list of examples for such variations is shown in \todo{add}.
The appearance of such variations then can be searched in the given reference strings.
Later in this subsection we will discuss the \gls{aho-corasick algorithm} which allows an efficient string matching.\\

Yet, such an approach has several disadvantages.
First, generating a list of name variations as part of an efficient search algorithm can be expensive in terms of memory usage.
The main reason for this is that names in reference strings can appear either as ``first name last name'' or as ``last name, fist name''.
Efficient string matching algorithms require the computation and storage of values for every variation that is generated.
For example, the \gls{rabin-karp algorithm} proposed by \citet{karp1987efficient} requires the generation and storage of hash values for every variation.
Generating the \gls{fsm} that is used in the \gls{aho-corasick algorithm} would result in a high redundancy of last name representations.
We will discuss this redundancy later in this subsection after introducing the algorithm.
In addition to a \gls{fsm}, the \gls{commentz-walter algorithm} \citep{commentz1979string} requires the computation and storage of shift values based on the variations similar to the \gls{knuth-morris-pratt algorithm} \citep{knuth1977fast}.
Generating and searching the variations of author names can thereby not be seen as a tractable approach for a high number of author names.
A way of preventing the generation of redundant variations is to first search for the first names and last names separately.
For example, \dots\todo{example}.\\

\citet{aho1975efficient} propose an efficient string matching algorithm that allows a matching of first and last names in a reference string.
This algorithm, called \gls{aho-corasick algorithm}, uses a \gls{fsm} that is generated from the dictionary entries to perform the matching.

The \gls{fsm} is constructed in the following way.

\itodo{explain state machine and its nodes and edges}
\itodo{example}
\itodo{statistics}

In order to perform the matching in only one iteration and without adding additional information to the \gls{fsm}, we can not decide whether a match is a first name or last name.
There are cases where a match can even be both a first name and a last name.
Example for such names are ``Arnold'', ``Ernst'', or ``Friedrich''.
We thereby tag a match with a generic ``name'' XML-tag and move this decision to the next step.\\

Given the reference strings where first names and last names are marked with a ``name'' tag, we now determine if a tagged name can be a first name, last name, or both.
For this we use a map that contains as a key the existing names and as value an array containing two Booleans to signal the name type.
If the lookup results in a pair of first name and last name, the names are tagged accordingly and a predefined number of surrounding words is saved.

After generating strings with possible first and last names including their context in a given reference string, we can perform a lookup of the tagged names in a dictionary of author names.
More concrete, the dictionary is a map containing the last names as key and a variation of possible first names as values.

\itodo{statistics}




\section{Training CRFs}\label{sec:ae-training-crfs}

