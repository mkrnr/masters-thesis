\chapter{Author Extraction}\label{cha:author-extraction}
In the following we will discuss the different steps during the author extraction process.

\section{Preprocessing}\label{sec:ae-preprocessing}

The corpus that was used in our study consists of \todo{number} research papers that are accessible via the \gls{ssoar}\footnote{\url{http://www.ssoar.info} (accessed April~21,~2016)} as \gls{pdf} documents.

Since our later steps require textual input, the first step is to extract the content from the \glspl{pdf}.
Apache PDFBox\footnote{\url{https://pdfbox.apache.org/} (accessed April~21,~2016)}, a Java library for manipulating \glspl{pdf}, allows the extraction of the content as Unicode text.
This is done by also taking into account the formatting of the document, for example when research papers contain two text columns of text.
\itodo{statistics about performance}
\itodo{reference section extraction using parscit}

\section{Generating Training Sets with Distant Supervision}\label{sec:ae-distant-supervision}

\subsection{Author Name Extraction}

In order to apply distant supervision to the task of labeling author names, an external source of information is needed.
For our evaluation we use two different sources.

\bigskip

The first one is based on the \acrfull{gnd}\footnote{\url{http://www.dnb.de/EN/Standardisierung/GND/gnd.html} (accessed April~27,~2016)} which is published by the German National Library in cooperation with other library networks and institutions.
The \gls{gnd}\todo{full acr?} contains information on persons, corporate bodies, conferences and other topics with a focus on the German-speaking world which includes Germany, Austria, and Switzerland.
\itodo{which persons are included, statistics}
We will refer to the second data set as the \gls{gnd} data set.

\bigskip

The second one was extracted from the Sociological Abstracts Module\todo{true?} which is part of the \gls{csa} Title Lists\footnote{\url{http://www.proquest.com/customer-care/title-lists/Title-Lists-CSA.html} (accessed April~27,~2016)} published by the company ProQuest\footnote{\url{http://www.proquest.com/about/} (accessed April~27,~2016)}.
\itodo{statistics, refernces}
We will refer to the second data set as the \gls{csa} data set.


\subsection{Context Extraction and Author Name Matching}
\itodo{variations of author names}
A crucial step for generating the training sets is the matching of author names in the extracted lists to their occurrences in the extracted reference sections.
One approach is to generate a list of possible ways how author names appear in a reference string.
A list of examples for such variations is shown in \todo{add}.
The appearance of such variations then can be searched in the given reference strings.

%TODO decisions:
% * first name = last name
% * possible overlaps of author names

%TODO statistics





\section{Training \glsentryshortpl{crf}}\label{sec:ae-training-crfs}

