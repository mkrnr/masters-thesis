\chapter{Distant Supervision}\label{cha:distant-supervision}

A common approach to learning \glspl{crf} is to use manually labeled data.
This results in an accurate labeling in most cases but is expensive and can thereby only be performed on small data sets.
The approach of \gls{distant supervision} that we discussed in \cref{cha:related-work} allows the automated labeling large data sets by using heuristics and external sources of information.
In this chapter we will first summarize how \gls{distant supervision} was used in the past. Based on this we will describe a way of using \gls{distant supervision} on the task of labeling authors in research papers. Finally we will discuss how this distantly supervised data set can be used as a training set for \glspl{crf}.\\

% TODO part of related work
%\section{Past Usages of Distant Supervision}
%\Gls{distant supervision} was first applied on relation extraction by \citet{mintz2009distant}.
%They used \gls{freebase} as a knowledge base for known relations in order to train their model.
%
%Instead of using a knowledge base, \citet{fan2015detecting} relied on a simple heuristic for localizing tables.
%They considered everything around a line starting with ``Table'' or ``Tab.'' potentially as a potential table.

\section{Distant Supervision in General}

\section{Training CRFs using Distant Supervision}
