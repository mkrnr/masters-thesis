\chapter{Distant Supervision}\label{cha:distant-supervision}

A common approach for training \glspl{crf} is to use manually labeled instances.
Manual labeling results in an accurate labeling in most cases but is expensive and can thereby only be performed on small data sets.

%TODO general words on distant supervision: jiang2012information

The approach of \gls{distant supervision} allows the automated labeling of large data sets by using heuristics and external sources of information.

In this chapter we will first give an overview of \gls{distant supervision} by discussing the past usages.
Based on this we will give a definition of the term.
We will then discuss approaches on how \glspl{crf} can be trained using distantly supervised data sets.



\section{Overview}

%TODO explain origin
Before explaining our usage of the term \gls{distant supervision}, we will first examine its origin and how it was previously used.

\bigskip

As discussed in \Cref{cha:related-work}, the term \gls{distant supervision} was introduced by \citet{mintz2009distant} as an approach for relation extraction without labeled data.
They state that \gls{distant supervision} extends the paradigm used by \citet{snow2005learning} for the extraction of hypernyms.
This is done by learning dependency paths from hypernym/hyponym word pairs that were extracted from WordNet~\citep{snow2005learning}.
The dependency paths are then used as features in a logistic regression classifier with the task of identifying hypernym pairs in a corpus~\citep{snow2005learning}.
Additionally, \citet{mintz2009distant} mention a similarity of \gls{distant supervision} to the usage of weakly labeled data in bioinformatics.
One mentioned example is how \citet{craven1999constructing} extract relations between biological objects such as proteins, cell-types, and diseases from a text corpus.
For this they train a Naive Bayes classifier with data from the Yeast Protein Database~\citep{payne1997yeast}.
\citet{surdeanu2012multi} go as far as saying that distant supervision for \gls{ie} was introduced by \citet{craven1999constructing}.

While not giving a definition of the term, \citet{mintz2009distant} state that ``[t]he intuition of distant supervision is that any sentence that contains a pair of entities that participate in a known Freebase relation is likely to express that relation in some way.''

There is now a body of research that uses the paradigm of \citet{mintz2009distant} for relation extraction~\citep{benson2011event,ritter2011named,nguyen2011end,takamatsu2012reducing,xu2013filling}.

\bigskip

Another intuition of \gls{distant supervision} that is not based on leveraging an existing knowledge base such as FreeBase.
\citet{go2009twitter} use emoticons in Twitter\footnote{\url{https://twitter.com/} (accessed May~19,~2016)} tweets as noisy labels to build the training set for sentiment analysis~\citep{go2009twitter}.

There is again a body of research that extends this approach to sentiment analysis~\citep{purver2012experimenting,marchetti2012learning,suttles2013distant}.

\citet{fan2015detecting} also apply distant supervision without using a knowledge base.
They rely on a simple heuristic for localizing tables by considering the context around a line starting with ``Table'' or ``Tab.'' as a potential table.

%TODO ling2012fine on entity recognition using distant supervision
\bigskip

Due to the different applications of \gls{distant supervision}, a precise definition\dots

What all approaches have in common is that they use some heuristic to assign labels to previously unlabeled data. These labelings are are used during the training, either in addition to preexisting labeled data or on their own.



\section{Distant Supervision and \glsentryshortpl{crf}}

\subsection{Marginalization}
%TODO \citet{tsuboi2008training}: problems with absolute input
%TODO \citet{tsuboi2008training}: marginalization of CRFs

\subsection{Generalized Expectation (\glsentryshort{ge})}
%TODO \citet{mann2007simple}, \citet{mann2008generalized}, \citet{mann2010generalized}: generalized expectation




