\chapter{Distant Supervision}\label{cha:distant-supervision}

A common approach to learning \glspl{crf} is to use manually labeled data.
This results in an accurate labeling in most cases but is expensive and can thereby only be performed on small data sets.
The approach of \gls{distant supervision} that we discussed in \cref{cha:related-work} allows the automated labeling large data sets by using heuristics and external sources of information.
In this chapter we will first summarize how \gls{distant supervision} was used in the past. Based on this we will describe a way of using \gls{distant supervision} on the task of labeling authors in research papers. Finally we will discuss how this distantly supervised data set can be used as a training set for \glspl{crf}.

\bigskip

% TODO part of related work
%\section{Past Usages of Distant Supervision}
%\Gls{distant supervision} was first applied on relation extraction by \citet{mintz2009distant}.
%They used \gls{freebase} as a knowledge base for known relations in order to train their model.
%
%Instead of using a knowledge base, \citet{fan2015detecting} relied on a simple heuristic for localizing tables.
%They considered everything around a line starting with ``Table'' or ``Tab.'' potentially as a potential table.

\section{Distant Supervision in General}
\section{Distantly Supervised Author Tagging}

In the following we will describe how distant supervision can be used to automatically tag author names in research papers.
For this we first extract author names from a structured data source.
Using this information we generate a number of variations to write the author names, for example with abbreviated first names.
This list of variations of author names then gets matched to the extracted text from our research papers.
%We will later use the matched data set to train \glspl{crf} (see \todo{ref}).

