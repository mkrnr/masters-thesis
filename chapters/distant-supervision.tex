\chapter{Distant Supervision}\label{cha:distant-supervision}

A common approach for training \glspl{crf} is to use manually labeled instances.
Manual labeling results in an accurate labeling in most cases but is expensive and can thereby only be performed on small data sets.

%TODO general words on distant supervision: jiang2012information

The approach of \gls{distant supervision} allows the automated labeling of large data sets by using heuristics and external sources of information.

In this chapter we will first give an overview of \gls{distant supervision} by discussing the past usages.
Based on this we will give a definition of the term.
We will then discuss approaches on how \glspl{crf} can be trained using distantly supervised data sets.



\section{Overview}

%TODO explain origin
Before explaining our usage of the term \gls{distant supervision}, we will first examine its origin and how it was previously used.

\bigskip

As discussed in \Cref{cha:related-work}, the term \gls{distant supervision} was introduced by \citet{mintz2009distant} as an approach for relation extraction without labeled data.
They state that \gls{distant supervision} extends the paradigm used by \citet{snow2005learning} for the extraction of hypernyms.
This is done by learning dependency paths from hypernym/hyponym word pairs that were extracted from WordNet~\citep{snow2005learning}.
The dependency paths are then used as features in a logistic regression classifier with the task of identifying hypernym pairs in a corpus~\citep{snow2005learning}.
Additionally, \citet{mintz2009distant} mention a similarity of \gls{distant supervision} to the usage of weakly labeled data in bioinformatics.
One mentioned example is how \citet{craven1999constructing} extract relations between biological objects such as proteins, cell-types, and diseases from a text corpus.
For this they train a Naive Bayes classifier with data from the Yeast Protein Database~\citep{payne1997yeast}.
\citet{surdeanu2012multi} go as far as saying that distant supervision for \gls{ie} was introduced by \citet{craven1999constructing}.

While not giving a definition of the term, \citet{mintz2009distant} state that ``[t]he intuition of distant supervision is that any sentence that contains a pair of entities that participate in a known Freebase relation is likely to express that relation in some way.''

There is now a body of research that uses the paradigm of \citet{mintz2009distant} for relation extraction~\citep{benson2011event,ritter2011named,nguyen2011end,takamatsu2012reducing,xu2013filling}.

\bigskip

Another intuition of \gls{distant supervision} that is not based on leveraging existing knowledge bases such as FreeBase.
\citet{go2009twitter} use emoticons in Twitter\footnote{\url{https://twitter.com/} (accessed May~19,~2016)} tweets as noisy labels to build the training set for sentiment analysis~\citep{go2009twitter}.

There is again a body of research that extends this approach to sentiment analysis~\citep{purver2012experimenting,marchetti2012learning,suttles2013distant}.

\citet{fan2015detecting} also apply distant supervision without using a knowledge base.
They rely on a simple heuristic for localizing tables by considering the context around a line starting with ``Table'' or ``Tab.'' as a potential table.

%TODO ling2012fine on entity recognition using distant supervision
\bigskip

Due to the different applications of \gls{distant supervision}, a precise definition\dots

What all approaches have in common is that they use some heuristic to assign labels to previously unlabeled data. These labelings are are used during the training, either in addition to preexisting labeled data or on their own.



\section{Distant Supervision and \glsentryshortpl{crf}}

Data sets that are used within a distantly supervised learning approach are typically incompletely annotated\todo{cite}.
This is due to the fact that external knowledge bases in practical scenarios do not cover all observed cases in the training set.
As a result, conventional \gls{crf} algorithms cannot be directly applied to such data sets since they require a fully annotated input~\citep{tsuboi2008training}.
In this section we will discuss approaches that apply \glspl{crf} on incompletely annotated data.
\itodo{sentence on tsuboi approach?}
A very promising approach, especially for our author extraction use case (see \Cref{cha:author-extraction}), is \gls{ge}, proposed by \citet{mann2007simple}.

\subsection{Marginalization}
%TODO \citet{tsuboi2008training}: problems with absolute input
%TODO instead: add the same prob distr. to all unknown tags?
\citet{tsuboi2008training} propose a method of training \glspl{crf} using incomplete annotations.
They consider two cases of incomplete annotations: Partial and ambiguous annotations.
By considering all possible label sequences that are with a given incomplete labeled sequence, it is possible to marginalize the unknown labels out.
Yet, the resulting equations is not a concave function which prevents efficient global maximization~\citep{tsuboi2008training}.
Instead, they rely on gradient ascent iterations which were proposed by~\citep{sha2003shallow}.
Another challenge that arises is the amount of label sequences that are generated this way which is exponential in the number of unknown labels~\citep{tsuboi2008training}.
This problem is addressed by applying the Markov assumption and using a modification of the Forward-Backward algorithm~\citep{tsuboi2008training}.
\citet{tsuboi2008training} apply their approach on Japanese word segmentation using partial annotations and on \gls{pos} tagging using ambiguous annotations.
For the word segmentation task the proposed approach was compared with two other methods that can handle partial annotations.
The first one is filling unknown labels by prediction based on the partial annotations and the second one is training a point-wise classifier that excludes label correlations~\citep{tsuboi2008training}.
The reported results show that the proposed method outperforms both methods for various sample sizes.
For the \gls{pos} tagging they used the Penn treebank corpus \citep{marcus1993building} which contains a number of ambiguous \gls{pos} tags.
The proposed approach was compared with three other heuristics, namely random selection, selecting the first tag in the description order, and selecting the most frequent tag in the corpus.
In this case, the proposed method only moderately improved the performance which the authors believe it due to the relatively low percentage of ambiguous tags in the corpus~\citep{tsuboi2008training}.

\subsection{Generalized Expectation (\glsentryshort{ge})}

Instead of ``filling the gaps'' of incomplete annotations, \citet{mann2008generalized} apply the concept of \gls{ge} on the training of linear-chain \glspl{crf}.
\Gls{ge} was first proposed in \citet{mann2007simple} under the name \gls{expectation regularization} as a method for semi-supervised learning.
%TODO definition GE

\citet{mann2008generalized} use \gls{ge} criteria to train \glspl{crf} on labeled features instead of fully labeled instances.
More precisely, the \gls{ge} criteria are used as an additional regularization term in the log-likelihood\todo{more precise} function during the inferencing task.
%TODO formula

\itodo{example}
One advantage is that it requires only the labeling of a number of occurrences of a feature in the data set instead of a full annotation.
Additionally, \gls{ge} criteria can take \glspl{conditional probability distribution} of labels given a feature as an input in order to train a model that matches this distributions \citep{mann2008generalized}.

\bigskip

%TODO objective function in mann2008 (and in mallet) is calculated with GE+gaussian prior instead of log-likelihood+gaussian prior+GE
