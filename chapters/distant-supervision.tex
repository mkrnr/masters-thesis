\chapter{Distant Supervision}\label{cha:distant-supervision}

% TODO Intro sentence

There are two ways to obtain a labeled data set for training the extraction of reference strings.
The first is to manually label data.
This results in an accurate labeling in most cases but is expensive and can thereby only be performed on small data sets.
The approach of \gls{distant supervision} that we discussed in \cref{cha:related-work} allows the automated labeling large data sets by using heuristics and external sources of information.
In this chapter we will discuss how \gls{distant supervision} was used in the past and how it can be applied on the task of extracting reference strings.\\

\Gls{distant supervision} was first applied on relation extraction by \citet{mintz2009distant}.
They used \gls{freebase} as a knowledge base for known relations in order to train their model.
Instead of using a knowledge base, \citet{fan2015detecting} relied on a simple heuristic for localizing tables.
They considered everything around a line starting with ``Table'' or ``Tab.'' potentially as a potential table.



% TODO Discussing the term itself (in related work section?!)

% TODO Explain options of how to apply it to our use case

%
