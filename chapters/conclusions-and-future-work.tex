\chapter{Conclusions and Future Work}\label{cha:conclusion-and-future-work}
\section{Conclusions}\label{sec:conclusion}
In this thesis, we presented an approach for the extraction of author names from reference sections of research papers from the area of German social sciences.
This includes the generation of distantly supervised training sets from unlabeled reference sections and the learning of a \gls{crf} model in combination with \gls{ge} constraints.

Further, we evaluated several aspects of the author extraction task using our approach.
One of them was the comparison of different author lists as the knowledge bases for distant supervision.
We demonstrated that tagging reference sections with an author set that comes from a similar research area does improve the performance of the resulting model.
Another aspect was the percentage of unmatched words that are used for generating \gls{ge} constraints.
Our evaluation shows that this percentage can be used to influence the typical trade-off between \gls{precision} and \gls{recall} of the resulting model.
Another outcome is that increasing the size of the training data set does improve the resulting model.
This is especially interesting since we do not rely on manually labeled data.


\section{Future Work}\label{sec:future-work}

\itodo{discuss challenges of distant supervision on our task (see reinanda2014prior)}
\itodo{sorting of PDFs into different categories (very well suited for extraction or not suited)}
\itodo{usage of available meta data about the paper (e.g.\ conference/journal title), e.g.\ to learn different models}
\itodo{add more labels and data from distant supervision, e.g.\ title, year, conference\dots}
\itodo{handling special case with dash in name, line break (consider Hans-Peter vs Mich-ael),--- or ``ders.'' instead of name,\dots}
\itodo{add organization names as authors}
\itodo{expand j in Generalized Expectation to include context (+ implementation)}
\itodo{implementation: storing of learned model}
\itodo{segmentation into reference strings: using year as anchor?\citep{powley2007high}}
\itodo{}
\itodo{evaluate impact of features (see \citep{bellare2007learning})}
\itodo{higher order crf to handle line breaks (currently kicked out)}



