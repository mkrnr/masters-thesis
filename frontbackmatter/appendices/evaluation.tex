\chapter{Evaluation}\label{app:cha-evaluation}
\section{Accuracy vs. F1 Score}\label{app:sec-accuracy-vs-f1-score}
In this section, we discuss the relation between the metrics \gls{accuracy} and \gls{f1 score} in the context of our evaluation.
As an example, we use the author extraction problem with the following five labels that follow the \gls{bio} format (see \Cref{sec:ae-building-ge-constraints}):
\begin{equation*}
  \mathit{Val}(Y_n)=\{\texttt{B-FN},\texttt{B-LN},\texttt{I-FN},\texttt{I-LN},\texttt{O}\}.
\end{equation*}

First, we consider $\mathit{TP}(L)$, $\mathit{FP}(L)$, $\mathit{TN}(L)$, and $\mathit{FN}(L)$.
As discussed in \Cref{cha:evaluation}, they refer to the number of True Positive, False Positive, True Negative, and False Negative assignments of the labels in a set of label $L$.

As an example, we have $L=\{\texttt{B-FN}\}$.
Based on this, the value of $\mathit{TP}(L)$ is the number of words that are correctly labeled with \texttt{B-FN} and $\mathit{FP}(L)$ is the number of words which are incorrectly labeled with \texttt{B-FN}.

For the cases $\mathit{TN}(L)$ and $\mathit{FN}(L)$, we now consider all labels that are not in $L$:
\begin{equation*}
  \mathit{NotL}=\mathit{Val}(Y_n)\setminus L.
\end{equation*}
In other words, when considering $L$ as the positive labeling, the labels in $\mathit{NotL}$ are considered negative labelings in $\mathit{TN}(L)$ and $\mathit{FN}(L)$.

For our example, we have $\mathit{NotL}=\mathit{Val}(Y_n)\setminus L=\{\texttt{B-LN},\texttt{I-FN},\texttt{I-LN},\texttt{O}\}$.
This results in $\mathit{TN}(L)$ being the number of words which are correctly labeled with one of the labels in $\mathit{NotL}$.
Further, $\mathit{FN}(L)$ is the number of words which are incorrectly labeled with one of the labels in $\mathit{NotL}$.

This also gives us for example $\mathit{TN}(L)=\mathit{TP}(\mathit{NotL})$.
Further, we have for example $\mathit{TN}(L)+\mathit{TP}(\mathit{NotL})=\mathit{TN}(L\cup\mathit{NotL})$.

Using this, we have:
\begin{equation*}
\begin{split}
  \mathit{accuracy}(L)&=\frac{TP(L)+TN(L)}{TP(L)+FP(L)+TN(L)+FN(L)}\\
  &=\frac{TP(L)+TP(\mathit{NotL})}{TP(L)+FP(L)+TP(\mathit{NotL})+FP(\mathit{NotL})}\\
  &=\frac{TP(L\cup\mathit{NotL})}{TP(L\cup\mathit{NotL})+FP(L\cup\mathit{NotL})}\\
  &=\mathit{precision}(L\cup\mathit{NotL}).\\
  &=\mathit{precision}(\textit{Val}(Y_n)).\\
\end{split}
\end{equation*}

Similarly, we can show that $\mathit{accuracy}(L)=\mathit{recall}(\textit{Val}(Y_n))$.
Since the \gls{f1 score} is the harmonic mean of \gls{precision} and \gls{recall}, we result in:
\begin{equation*}
\begin{split}
  \mathit{accuracy}(L)&=\textit{F1}(\textit{Val}(Y_n)).\\
\end{split}
\end{equation*}

\section{Configuration}\label{app:sec-configuration}

