\begin{center}
  \begin{large}
  \bfseries{Zusammenfassung}
  \end{large}
\end{center}
Zusammenfassung auf Deutsch
\par\bigskip
\par\bigskip
\selectlanguage{english}
\begin{center}
  \begin{large}
  \bfseries{Abstract}
  \end{large}
\end{center}
To help in overcoming a shortage of citation information for the German social sciences, we contribute an approach for extracting author names from reference sections.
Instead of relying on small amounts of manually labeled data, we use a distantly supervised approach to automatically generate a partially labeled training data set.
To apply this data set to the widely used probabilistic framework of conditional random fields, GE criteria provide a suitable objective function.
The resulting model does not only decide if a word is part of an author, but also separates the listed authors and distinguishes between first and last names.
The evaluation of our model reports a promising performance for the author extraction task.
In addition, it suggests ways of influencing the trade-off between the precision and recall of the model.
